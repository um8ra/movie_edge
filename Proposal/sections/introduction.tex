It's Friday night and you would like to watch a movie.  You turn on your SmartTV, open \href{http://www.netflix.com}{Netflix}, and as you scroll through the \textbf{Recommended for You} list your frustration grows.   \textit{How did this get on the list?  I'm not interested in that tonight.  I just saw that film and hated it!}  

We have all been there.   While recommendation engines have grown in sophistication and popularity, they are black-box in nature and often fall short in satisfying the immediate need of users. This frustration has serious implications for streaming services such as Netflix \& Hulu, who wish to maintain market share and ongoing subscription revenue.  

\subsection{What are we trying to do?}

In 2009 a \$1mm Grand Prize was awarded to the winner of \href{https://www.netflixprize.com/}{The Netflix Prize}. The contest, aimed at improving their internal CinematchSM recommendation system, was captured by a group including scientists from AT\&T Labs \cite{koren2009bellkor}.  Their innovative collaborative filtering system was able to demonstrate a 10\% improvement over Netflix.

And yet, 10 years later, collaborative filters and recommendation engines suffer the same comprehension challenge as many of today's machine learning systems.  (TODO: reference something) The results come without the necessary context to explain why a particular movie is the one to pick.

We propose MovieDelver, our interactive movie recommendation platform, which puts the users in charge of the dials and switches that drive the recommendations.   MovieDeliver will be data driven, leveraging  \href{https://www.omdbapi.com/}{The Open Movie Database (OMDd) API} and the movielens [cite!] rating data.   And MovieDelver will be interactive \& visual: as users refine search criteria recommendations will be updated and displayed in real-time.

MovieDelver extends MovieExplorer \cite(taijala2018movieexplorer). MovieExplorer is a Collaborative-Filtering (CF) based recommender system (RS) that offers guided exploration of the latent space induced by a CF model. Where CF engines typically learn long-term user preferences, MovieExplorer's interface allows users to specify what they are looking for right now, even if this is different from what they usually watch. MovieDelver will enhance MovieExplorer's latent space exploration my locally projecting search directions into an alternative feature space that is much more human-interpretable, in the spirit of LIME \cite(ribeiro2016should)