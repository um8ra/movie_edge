\subsection{Risks and Payoffs?}

The primary risk this project poses is that it may be too agressive in too short of a timeline. We are planning on incorporating many facets learned over the course of our time at Georgia Tech. These include working with graph data, self-teaching machine learning (specifically neural networks), similarity analysis, web development and visualization. Our group views our project as being quite sizable to accomplish. Computational costs may also be a factor since neural nets take a lot time to train, and we have hard deadlines to hit.

Knowing the risks, the rewards are high: we may have a movie recommendation engine that allows us more detail and granularity than Netflix does, and certainly with more transparency since we'll have both the source data and source code. Previously, Netflix allowed people to feed data on a five star scale. Shortly after deep learning started taking off, they changed this to "thumbs up" or "thumbs down," a binary choice lacking granularity. We surmise that this is because neural network embeddings are based on one-hot encoded vectors and it was more efficient at Netflix's scale to utilize embeddings than a traditional feed-forward neural network.

Our approach will, at a high computational cost, allow us to specify movie ratings with greater granularity and capitalize on that accuracy in that the output will also be more accurate (we hope). This has the payoff, we hope, of allowing us to predict movies with greater accuracy. Upon finding similar movies and their projected ratings based on user input, a user could even find other users with similar taste and explore their likes and dislikes. In essence, we are allowing for two outcomes: movie rating prediction, and user similarity prediction. Seeing as humans still are the most computationally powerful neural nets on the planet, this "user exploration" may be a very powerful approach to recommendations.

While our end users may be a bunch of computer scientists for the moment, handing someone a vector and stating with excitement that "here are your recommendations, how exciting!" is not likely to yield an ecstatic customer. We will visualize the output to make our end product more usable by people that aren't CS or Analytics grad students.
