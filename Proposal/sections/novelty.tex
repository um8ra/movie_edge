\subsection{Novelty}
What's new in your approach? Why will it be successful?

As discussed, this project extends MovieExplorer \cite{taijala2018movieexplorer}. We do so in two main ways. 

First, MovieDelver replaces the traditional matrix factorization based CF model in MovieExplorer with a state of the art  CF model based on Word2Vec \cite{mikolov2013efficient}, \cite{rong2014word2vec}, and \cite{rong2014word2vec}. In Natural Language Processing applications, identifying the word context in an embedding vector space via representation learning has been widely adopted. [JT: Can YS describe key ideas from these 3 papers here or elsewhere?] To our knowledge, modern embedding techniques have not been applied to recommendation engines, although the now-traditional matrix factorization methods can be seen as a naive form of embedding. \st{, where activation functions are linear. We can treat movies as words and users as sentences to establish an embedding problem.} One significant issue with embedding approaches is that the neural networks are highly parameterised and thus extremely difficult to interpret.

Second, where MovieExplorer allows users to guide the recommendation search in the latent CF model space by taking steps "towards" or "away" from movies, we believe that we can improve interpretability of these steps in the CF model space by allowing exploration in terms of other movie features (genre, user tags, directors, actors) instead. \st{?? This may increase recommendation novelty and promote serendipity.} The key to this extension is to leverage LIME  \cite{ribeiro2016model} or  \cite{ribeiro2016should}. LIME makes machine learning models interpretable by sampling around a desired observation and building a surrogate locally faithful linear model that is used for explanation. We will build a LIME explainer using human-interpretable movie features and use user feedback on the LIME explainer to refine MovieDelver's suggestions. LIME has been recently adapted to CF tasks \cite{nobrega2019towards} as LIME-RS. LIME-RS extends LIME by adding side-channel features as well as improving LIME's sampling strategy which may not be suitable given the sparsity of the CF input space.

	