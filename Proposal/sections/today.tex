\subsection{How is it done today}
what are the limits of current practice?
[JON - This can be edited down ALOT]
Recommendation engines are ubiquitous in the modern digital economy. However, they are typically designed to be unobtrusive and work behind the scenes. Even Netflix, with its famous investment in the 2007 Netflix prize, does not heavily advertise its engine when presenting recommended movies or shows. Most industrial recommender engines rely on relatively simple models. For example, while it adopted some novel approaches from the Netflix Prize, Netflix did not implement the \href{https://www.wired.com/2012/04/netflix-prize-costs/}{winning solution} Koren 2009 \cite{koren2009bellkor}, citing engineering costs. Similarly, Amazon continues to rely on a system rooted in item-to-item collaborative filtering \cite{smith2017two}, one of the simplest early approaches that scales by performing expensive calculations offline and focuses on finding similar items instead of similar customers \cite{linden2003amazon}. In the meantime, increasingly sophisticated recommendation algorithms have been developed in academia. For example, Koren 2008 \cite{koren2008factorization} extends the matrix factorization approach to movie recommendation pioneered \cite{funk2006netflix} in the Netflix Prize by combining matrix factorization with item-to-item collaborative filtering techniques, taking advantage of implicit feedback on non-rated movies inferred from user behavior. Rendle 2012 \cite{rendle2012factorization} developed the factorization machine, which generalizes previous matrix factorization techniqus, which were limited to 2nd order interactions, to higher order interactions while maintaining computational efficiency.

Ironically, while all the approaches described above are interpretable in some sense, most commercial implementations do not appear to leverage that interpretability to a large extent by allowing users to explicitly control how recommendations are being generated. This seems to be for two reasons. First, companies seem to want to keep the recommendation engine out of the limelight while surfacing the recommendations themselves. Second, matrix factorization recommendation produces latent factors that can be hard for humans to interpret, even if the total effect is a simple linear sum of components. That is to say, while the the recommendation is expressed in an easy to understand model, is embedded in a space that humans cannot easily interpret. We believe that not accounting for users' input at the recommendaton stage is a disservice - users should be able to tell the engine what they are looking for rather than having that information be imputed from their previous interactions with the system. 
