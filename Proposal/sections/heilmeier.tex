\subsection{What are you trying to do?} 

MovieEdge is an accurate movie RS that allows for interactive exploration of the models’ taste space. 

\subsection{How is it done today?}

Commercial RS does not incorporate interactive user feedback. Recent academic approaches can incorporate feedback but do not have a visualization layer nor provide any supplemental information to users.  

\subsection{What is New?}

MovieEdge builds upon ideas from two recent systems - MovieExplorer and Embedding Projector - and will combine an accurate state of the art RS, interactive exploration of the taste space, and visualization of the exploration process.  

\subsection{Who Cares?}

Incorporating user feedback in RS has been shown to improve user satisfaction. Adding a visualization aspect to the exploration process will increase the appeal of a RS. 

\subsection{Difference and Impact?}

RS drives a large portion of eCommerce activity. Improving user satisfaction can be a competitive advantage for companies. The visualization and interactive exploration aspect may increase serendipity, which is a desired trait of RS. Time permitting, a user satisfaction survey will be administered to quantify the benefits. 

\subsection{Risks and Payoffs?}

MovieEdge is being developed on a widely available anonymized research dataset, so privacy and licensing risks are minimal. Well-tested libraries for CF and visualization exist and will be used to de-risk the implementation. 

The payoffs are to build a better interface for users to explore and interact with CF models thus increasing satisfaction. 

The primary project risk is scope: have we taken on too much?   We minimize this risk through parallel and iterative development focused on  incremental improvements that validate our approach.

\subsection{Cost?}
We may have to pay to pre-compute and host our models. These costs should be under \$100.

\subsection{How Long?}
We plan to leverage off the shelf software libraries as much as possible and adapt code where necessary. The work is deadline-driven.  A “Midterm exam” is summarized by the “Coding Work” + “Progress Report” sections; “Final exam” is the “Algorithm Polish” and “Final Report” section.  See Figure \ref{fig:projectplan}  (Page \pageref{fig:projectplan}) for details.  All team members have and will continue to contribute equally.

\subsection{Milestones?}
We have gathered the data from MovieLens \cite{harper2016movielens} and the Open Movie Database \cite{openMovieDB}. Our mid-point check in is to build a working RS and implement exploration of the taste space. Our final goal is to complete the visualization and interactive elements of MovieEdge.

