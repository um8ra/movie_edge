%%
%%
%% The first command in your LaTeX source must be the \documentclass command.
\documentclass[sigchi, 12pt, nonacm=true, timestamp=true, screen=true]{acmart}

%%
%% \BibTeX command to typeset BibTeX logo in the docs
\AtBeginDocument{%
  \providecommand\BibTeX{{%
    \normalfont B\kern-0.5em{\scshape i\kern-0.25em b}\kern-0.8em\TeX}}}

%% Rights management information.  This information is sent to you
%% when you complete the rights form.  These commands have SAMPLE
%% values in them; it is your responsibility as an author to replace
%% the commands and values with those provided to you when you
%% complete the rights form.
%%\setcopyright{acmcopyright}
%%\copyrightyear{2018}
%%\acmYear{2018}
%%\acmDOI{10.1145/1122445.1122456}

%% These commands are for a PROCEEDINGS abstract or paper.
%%\acmConference[Woodstock '18]{Woodstock '18: ACM Symposium on Neural
%%  Gaze Detection}{June 03--05, 2018}{Woodstock, NY}
%%\acmBooktitle{Woodstock '18: ACM Symposium on Neural Gaze Detection,
%%  June 03--05, 2018, Woodstock, NY}
%%\acmPrice{15.00}
%%\acmISBN{978-1-4503-9999-9/18/06}


%%
%% Submission ID.
%% Use this when submitting an article to a sponsored event. You'll
%% receive a unique submission ID from the organizers
%% of the event, and this ID should be used as the parameter to this command.
%%\acmSubmissionID{123-A56-BU3}

%%
%% The majority of ACM publications use numbered citations and
%% references.  The command \citestyle{authoryear} switches to the
%% "author year" style.
%%
%% If you are preparing content for an event
%% sponsored by ACM SIGGRAPH, you must use the "author year" style of
%% citations and references.
%% Uncommenting
%% the next command will enable that style.
%%\citestyle{acmauthoryear}

\setcopyright{none}

\usepackage{hyperref}
\usepackage{subfiles}

%%
%% end of the preamble, start of the body of the document source.
\begin{document}

%%
%% The "title" command has an optional parameter,
%% allowing the author to define a "short title" to be used in page headers.
\title{Team 5 Project Proposal - Movie Mood}

%%
%% The "author" command and its associated commands are used to define
%% the authors and their affiliations.
%% Of note is the shared affiliation of the first two authors, and the
%% "authornote" and "authornotemark" commands
%% used to denote shared contribution to the research.

\author{Rocko Graziano}
\email{rpgraziano@gatech.edu}
\affiliation{%
  \institution{Georgia Tech OMSCS}
}
\author{Daniel Klass}
\email{dklass3@gatech.edu}
\affiliation{%
	\institution{Georgia Tech OMSCS}
}
\author{Yi Sun}
\email{ysun428@gatech.edu}
\affiliation{%
	\institution{Georgia Tech OMSCS}
}
\author{Jonathan Tay}
\email{jtay6@gatech.edu}
\affiliation{%
	\institution{Georgia Tech OMSCS}
}


%%
%% By default, the full list of authors will be used in the page
%% headers. Often, this list is too long, and will overlap
%% other information printed in the page headers. This command allows
%% the author to define a more concise list
%% of authors' names for this purpose.
%\renewcommand{\shortauthors}{Trovato and Tobin, et al.}

%%
%% The abstract is a short summary of the work to be presented in the
%% article.
\begin{abstract}
We present Movie Mood, an interactive recommendation engine which allows users to both understand why a movie is recommended and tune those recommendations for their current viewing interest.
\end{abstract}

%%
%% The code below is generated by the tool at http://dl.acm.org/ccs.cfm.
%% Please copy and paste the code instead of the example below.
%%
%%\begin{CCSXML}
%%<ccs2012>
%% <concept>
%%  <concept_id>10010520.10010553.10010562</concept_id>
%%  <concept_desc>Computer systems organization~Embedded systems</concept_desc>
%%  <concept_significance>500</concept_significance>
%% </concept>
%% <concept>
%%  <concept_id>10010520.10010575.10010755</concept_id>
%%  <concept_desc>Computer systems organization~Redundancy</concept_desc>
%%  <concept_significance>300</concept_significance>
%% </concept>
%% <concept>
%%  <concept_id>10010520.10010553.10010554</concept_id>
%%  <concept_desc>Computer systems organization~Robotics</concept_desc>
%%  <concept_significance>100</concept_significance>
%% </concept>
%% <concept>
%%  <concept_id>10003033.10003083.10003095</concept_id>
%%  <concept_desc>Networks~Network reliability</concept_desc>
%%  <concept_significance>100</concept_significance>
%% </concept>
%%</ccs2012>
%%\end{CCSXML}

%%\ccsdesc[500]{Computer systems organization~Embedded systems}
%%\ccsdesc[300]{Computer systems organization~Redundancy}
%%\ccsdesc{Computer systems organization~Robotics}
%%\ccsdesc[100]{Networks~Network reliability}

%%
%% Keywords. The author(s) should pick words that accurately describe
%% the work being presented. Separate the keywords with commas.
%%\keywords{movies, DVA, lorem ipso}

%% A "teaser" image appears between the author and affiliation
%% information and the body of the document, and typically spans the
%% page.
%%\begin{teaserfigure}
%%  \includegraphics[width=\textwidth]{sampleteaser}
%%  \caption{Seattle Mariners at Spring Training, 2010.}
%%  \Description{Enjoying the baseball game from the third-base
%%  seats. Ichiro Suzuki preparing to bat.}
%%  \label{fig:teaser}
%%\end{teaserfigure}

%%
%% This command processes the author and affiliation and title
%% information and builds the first part of the formatted document.
\maketitle

\section{Introduction}
\subfile{sections/introduction}

\section{Project Overview}

\subfile{sections/today}

\subfile{sections/novelty}

\subfile{sections/audience}

\subfile{sections/difference-impact}

\subfile{sections/risks-payoff}

The primary risk this project poses is that it may be too agressive in too short of a timeline. We are planning on incorporating many facets learned over the course of our time at Georgia Tech. These include working with graph data, self-teaching machine learning (specifically neural networks), similarity analysis, web development and visualization. Our group views our project as being quite sizable to accomplish. Computational costs may also be a factor since neural nets take a lot time to train, and we have hard deadlines to hit.

Knowing the risks, the rewards are high: we may have a movie recommendation engine that allows us more detail and granularity than Netflix does, and certainly with more transparency since we'll have both the source data and source code. Previously, Netflix allowed people to feed data on a five star scale. Shortly after deep learning started taking off, they changed this to "thumbs up" or "thumbs down," a binary choice lacking granularity. We surmise that this is because neural network embeddings are based on one-hot encoded vectors and it was more efficient at Netflix's scale to utilize embeddings than a traditional feed-forward neural network.

Our approach will, at a high computational cost, allow us to specify movie ratings with greater granularity and capitalize on that accuracy in that the output will also be more accurate (we hope). This has the payoff, we hope, of allowing us to predict movies with greater accuracy. Upon finding similar movies and their projected ratings based on user input, a user could even find other users with similar taste and explore their likes and dislikes. In essence, we are allowing for two outcomes: movie rating prediction, and user similarity prediction. Seeing as humans still are the most computationally powerful neural nets on the planet, this "user exploration" may be a very powerful approach to recommendations.

While our end users may be a bunch of computer scientists for the moment, handing someone a vector and stating with excitement that "here are your recommendations, how exciting!" is not likely to yield an ecstatic customer. We will visualize the output to make our end product more usable by people that aren't CS or Analytics grad students.

\subfile{sections/costs}	

\subfile{sections/projectplan}
Progress and milestones further into the future may not have individuals assigned yet and are highly tentative. To quote Helmuth von Moltke the Elder: "No plan survives contact with the enemy."

\begin{itemize}
  \item Get the data (Due: October 11)
    \begin{itemize}
      \item Download Movielens data (direct download, all members, individually) 
      \item Download OMDB data (API download, split among members equally in parallel)
    \end{itemize}
  \item Turn in proposal (Due: October 11)
    \begin{itemize}
      \item Intro: Rocko
      \item How it is done today: Yi
      \item Novelty: Yi, Jonathan
      \item Who cares: Rocko
      \item Difference & Impact: Jonathan, Yi
      \item Risks and payoffs: Daniel
      \item Costs: TBD
      \item Project Plan and Milestones: Daniel
      \item Conclusion: Rocko
      \item Bibliography: Rocko
    \end{itemize}
  \item Train ML algorithms on Movielens dataset and output model (Due: October 28)
    \begin{itemize}
      \item Research: All (75% done, always finding new things to do)
      \item Development: Daniel (50% done)
      \item Polish Code: All (0% done)
      \item Deploy to cloud for training?: All (0% done)
    \end{itemize}
  \item Create web API so that users can input their data (HTML/JS) and the ML model can interpret it (Python) (Due November 5)
    \begin{itemize}
      \item Create webpage with form to submit movie ratings
      \item Create view to receive form submission and render output (discuss if we want to persist these in a database, difficulty++)
      \item Reply to submission with ML output
    \end {itemize}
  \item Progress Report
    \begin{itemize}
      \item Proposed method
      \item Design of upcoming experiments / evaluation
      \item Plan of activities (likely Daniel)
      \item List of innovations
      \item Submission (nomination of submitter TBD)
      \item Distribution of team member effort (strive for equality)
    \end{itemize}
  \item Vizualize interpretation (Due November 12)
    \begin{itemize}
      \item Solid lines for same director or actors?
      \item Dotted line for other reason(s) TBD?
      \item Colorize by estimated rating for the end user
    \end{itemize}
  \item Add filters on data based on OMDB data (this is our interactive portion) (Due November 19)
    \begin{itemize}
      \item Director(s)
      \item Actor(s)
      \item Studio? E.g. Pixar movies; Studio Ghibli;
      \item Producer? Probably not. Producer == moneybags not creative.
    \end{itemize}
  \item $$$
  \item Profit
\end{itemize}

\subfile{sections/conclusion}


%% the bibliography file.
\bibliographystyle{ACM-Reference-Format}
\bibliography{sections/bibliography}

\end{document}
\endinput

%%
%% End of file `sample-sigconf.tex'.
